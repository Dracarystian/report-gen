\documentclass{beamer}
\usepackage[utf8]{inputenc}
\usepackage[spanish]{babel}
\usepackage{lmodern}
\usepackage[T1]{fontenc}

\title{Informe Ejecutivo sobre Proyectos de Ley Relevantes}
\author{Report Generator AI}
\date{\today}

\begin{document}

\begin{frame}
    \titlepage
\end{frame}

\begin{frame}{Resumen Ejecutivo}
Este informe proporciona una visión detallada de los proyectos de ley más relevantes, destacando las tendencias y patrones observados en la base de datos.\\

Se enfoca en la actividad legislativa reciente, analizando las principales iniciativas, su estado actual y los actores involucrados. Este análisis es crucial para entender el panorama legislativo y su impacto potencial en diversos sectores.
\end{frame}

\begin{frame}{Proyectos de Ley Destacados (1/2)}
\textbf{1. Reforma Pensional}
\begin{itemize}
    \item \textbf{Estado Actual}: La reforma avanza en el Congreso, habiendo superado el segundo debate en el Senado. Se han aprobado modificaciones significativas, como el ajuste del umbral de cotización en Colpensiones a 2,3 salarios mínimos.
    \item \textbf{Impacto Esperado}: Busca ampliar el acceso a la seguridad social y mejorar las condiciones de retiro para más ciudadanos. Sin embargo, enfrenta desafíos para su implementación, con propuestas para retrasar su entrada en vigor.
\end{itemize}

\vspace{1ex}
\textbf{2. Open Finance en el Plan Nacional de Desarrollo}
\begin{itemize}
    \item \textbf{Progreso}: Se espera que los artículos 71 y 75 del proyecto de ley impulsen la apertura de datos financieros, fomentando la competencia y la inclusión financiera.
    \item \textbf{Desafíos}: La reglamentación y la seguridad de los datos son áreas críticas que requieren atención para garantizar el éxito del Open Finance.
\end{itemize}
\end{frame}

\begin{frame}{Proyectos de Ley Destacados (2/2)}
\textbf{3. Ley de Creación del Ministerio de Igualdad}
\begin{itemize}
    \item \textbf{Objetivo}: Establecer el Ministerio de Igualdad y Equidad, con un enfoque en la estructura orgánica y la adopción de políticas inclusivas.
    \item \textbf{Estado}: En fase de discusión, con seguimiento activo por parte de actores clave.
\end{itemize}
\end{frame}

\begin{frame}{Tendencias y Patrones}
\begin{itemize}
    \item \textbf{Aumento en la Actividad Legislativa}: Se observa un incremento en la radicación de proyectos de ley, especialmente aquellos relacionados con la inclusión social y la equidad.
    \item \textbf{Enfoque en Sostenibilidad y Equidad}: Muchas iniciativas buscan abordar temas de sostenibilidad, igualdad y equidad, reflejando una tendencia hacia políticas más inclusivas y sostenibles.
\end{itemize}
\end{frame}

\begin{frame}{Actores Clave}
\begin{itemize}
    \item \textbf{Ministerio de Hacienda}: Juega un papel crucial en la implementación de políticas económicas y financieras, como se observa en las reformas propuestas.
    \item \textbf{Congreso de la República}: Activo en la discusión y aprobación de leyes que impactan directamente en la estructura social y económica del país.
\end{itemize}
\end{frame}

\begin{frame}{Conclusiones}
\begin{itemize}
    \item \textbf{Progreso en Reformas Clave}: Las reformas pensional y de Open Finance son ejemplos de iniciativas que pueden transformar significativamente el panorama económico y social.
    \item \textbf{Desafíos Persistentes}: La implementación efectiva de estas reformas enfrenta obstáculos, principalmente en términos de reglamentación y aceptación política.
\end{itemize}
\end{frame}

\begin{frame}{Recomendaciones}
\begin{itemize}
    \item \textbf{Monitoreo Continuo}: Es crucial seguir de cerca el progreso de estas iniciativas para anticipar cambios regulatorios y ajustar estrategias en consecuencia.
    \item \textbf{Participación Activa}: Los actores interesados deben involucrarse activamente en el proceso legislativo para influir en el resultado de las reformas y asegurar que sus intereses sean considerados.
\end{itemize}
\end{frame}

\begin{frame}{Formato Profesional}
Este informe está diseñado para proporcionar una visión clara y concisa de los proyectos de ley más relevantes, utilizando un formato estructurado que facilita la comprensión y el análisis de la información presentada.
\end{frame}

\end{document}
