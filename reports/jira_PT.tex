
\documentclass{article}
\usepackage[utf8]{inputenc}
\usepackage{graphicx}
\usepackage{hyperref}
\usepackage{geometry}
\usepackage{booktabs}

\title{Informe Ejecutivo de Jira - Procesos TI}
\author{Report Generator AI}
\date{\today}

\begin{document}

\maketitle
\tableofcontents

\section{Informe Ejecutivo de Jira - Procesos TI  }
\textbf{Período de análisis: Enero - Febrero 2025}

---

#\section{1. Resumen Ejecutivo}

Durante el periodo analizado, el equipo de TI gestionó y finalizó un total de 33 issues en Jira, abarcando tanto historias de usuario como tareas técnicas. Todas las iniciativas registradas fueron completadas, evidenciando un alto nivel de cumplimiento y eficiencia en la ejecución de los procesos. Las actividades se centraron en mejoras funcionales, ajustes de interfaz, integraciones técnicas y optimización de flujos internos para la plataforma DapperLatam, con una distribución equilibrada de responsabilidades entre los principales integrantes del equipo.

---

#\section{2. Análisis de los Datos Principales}

##\section{2.1. Estado de los Issues}

| Estado      | Total |
|-------------|-------|
| Finalizada  | 33    |

\textbf{Observación:}  
El 100% de los issues reportados durante el periodo fueron finalizados, lo que refleja un óptimo desempeño del equipo y una adecuada planificación de los sprints.

---

##\section{2.2. Distribución por Asignado}

| Asignado                                 | Total de Issues |
|------------------------------------------|-----------------|
| Diego Ramirez                            | 14              |
| David Sebastian Castiblanco Velasquez    | 13              |
| Andrés Laverde                           | 4               |
| Sin asignar                              | 2               |

\textbf{Observación:}  
La asignación de tareas se distribuyó principalmente entre Diego Ramirez y David Sebastian Castiblanco Velasquez, quienes lideraron la ejecución técnica y funcional de los requerimientos.

---

##\section{2.3. Distribución por Tipo}

| Tipo      | Total |
|-----------|-------|
| Tarea     | 24    |
| Historia  | 9     |

\textbf{Observación:}  
La mayoría de los issues corresponden a tareas técnicas (72%), mientras que las historias de usuario representaron el 28%, lo que indica un enfoque en la implementación y ajustes técnicos, así como en la mejora continua de la experiencia del usuario.

---

##\section{2.4. Etiquetas Utilizadas}

| Etiqueta | Total |
|----------|-------|
| Back     | 2     |
| Front    | 1     |

\textbf{Observación:}  
El uso de etiquetas fue bajo, lo que puede dificultar la categorización y posterior análisis temático de los issues.

---

##\section{2.5. Principales Temáticas Abordadas}

- \textbf{Mejoras en la visualización y gestión de transcripciones e informes legislativos.}
- \textbf{Optimización de formularios y flujos de registro.}
- \textbf{Ajustes en la segmentación y presentación de indicadores económicos y noticias.}
- \textbf{Integraciones técnicas (API, endpoints, notificaciones WhatsApp, Twilio).}
- \textbf{Actualizaciones de interfaz y cambios de copy para mayor claridad y usabilidad.}
- \textbf{Gestión y visualización de actores de gobierno y proyectos de ley relacionados.}

---

#\section{3. Conclusiones}

- El equipo logró finalizar la totalidad de los issues planificados, evidenciando una gestión eficiente y cumplimiento de objetivos.
- La carga de trabajo estuvo equilibrada entre los principales desarrolladores, lo que favorece la continuidad y calidad de los entregables.
- El enfoque predominante fue técnico, con múltiples tareas orientadas a la mejora de procesos internos y la integración de nuevas funcionalidades.
- Se identificó un uso limitado de etiquetas, lo que podría dificultar la trazabilidad y análisis a largo plazo.

---

#\section{4. Recomendaciones}

1. \textbf{Fomentar el uso de etiquetas:}  
   Incrementar la aplicación de etiquetas temáticas y técnicas para facilitar la clasificación, seguimiento y análisis de los issues en futuras iteraciones.

2. \textbf{Documentar criterios de aceptación y descripciones:}  
   Asegurar que todas las tareas e historias cuenten con descripciones y criterios de aceptación claros, para mejorar la comunicación y comprensión entre los miembros del equipo.

3. \textbf{Mantener el equilibrio en la asignación de tareas:}  
   Continuar con la distribución equitativa de responsabilidades para evitar la sobrecarga de trabajo y garantizar la sostenibilidad del rendimiento del equipo.

4. \textbf{Monitorear la evolución de los procesos:}  
   Implementar revisiones periódicas para identificar oportunidades de mejora en la gestión de issues y la optimización de flujos de trabajo.

---

\textbf{Elaborado por:}  
Analista de Procesos TI  
Fecha: 14 de marzo de 2025

---

\end{document}
    